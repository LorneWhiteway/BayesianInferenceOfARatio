% http://www.ctan.org/tex-archive/macros/latex/contrib/beamer/examples
% http://latex.artikel-namsu.de/english/beamer-examples.html

%\documentclass{beamer}
\documentclass[usenames,dvipsnames]{beamer}
\usepackage{amsmath}
\usepackage{amssymb}
\usepackage{bm}
\usepackage{fancybox, graphicx}
\usepackage{listings}
\usepackage{tikz} % Diagrams
\usepackage{color}
\usepackage{textcomp} % See https://tex.stackexchange.com/questions/145416/how-to-have-straight-single-quotes-in-lstlistings

\lstset{language=bash,upquote=true} % Format listings as appropriate for bash. Inexplicably we get problems if the language is set as part of the \begin{lstlisting} command.

% https://tex.stackexchange.com/questions/36030/how-to-make-a-single-word-look-as-some-code
\definecolor{light-gray}{gray}{0.95}
\newcommand{\code}[1]{\colorbox{light-gray}{\texttt{#1}}}

\usetheme{boxes}
\usecolortheme{beaver}

\title{What Can a Bayesian Say About $y/x$?}
\author{Lorne Whiteway \\ lorne.whiteway.13@ucl.ac.uk}
\institute{Astrophysics Group \\ Department of Physics and Astronomy \\ University College London}
\date{21 November 2025}

\begin{document}

\frame{\titlepage}

\begin{frame}{Purpose of presentation}
  \begin{block}{}
    \begin{itemize}
      \item{I'll show how to do Bayesian inference in a context that's simple but non-trivial.}
      \item{Talk will be pedagogical and mostly mathematical.}
      \item{Prior knowledge of Bayesian ideas is useful but not necessary.}
      \item{Thanks to Niall Jeffrey for help.}
    \end{itemize}
  \end{block}
\end{frame}

\begin{frame}{The Original Cosmological Motivation (1)}
  \begin{block}{}
    \begin{itemize}
      \item{The \textit{overdensity} of dark matter (DM), denoted $\delta_{\textrm{DM}}$,  is the percentage difference between the local density (in some volume) and the universal average density. }
      \item{Overdensities have existed since early times. Gravity draws DM into overdense regions, so the overdensities get bigger with time.}
      \item{There is a similar definition for baryons. Baryons fall into DM overdensities, so we get baryon overdensities in the same place.}
    \end{itemize}
  \end{block}
\end{frame}

\begin{frame}{The Original Cosmological Motivation (2)}
  \begin{block}{}
    \begin{itemize}
      \item{Physical effects mean that the baryon overdensity need not equal the DM overdensity; a simple model assumes a linear relationship:
      \begin{equation*}
      b = \frac{\delta_{\textrm{baryon}}}{\delta_{\textrm{DM}}}
      \end{equation*}
      }
      \medskip
      \item{In a recent paper (\url{https://arxiv.org/abs/2509.18967}), we (Ellen, Niall, LW, Ofer, Josh, et al.) measured the baryon overdensity (from galaxy counts) and the DM overdensity (from weak-lensing mass maps). But how then to infer something about $b$?}
    \end{itemize}
  \end{block}
\end{frame}

\begin{frame}{Restate as a Statistics Problem}
  \begin{block}{}
    \begin{itemize}
      \item{We make noisy measurements of $x$ and $y$ and we want to infer $b = y/x$.}
      \item{The answer should be probabilistic (due to the uncertainty arising from the measurement noise).}
    \end{itemize}
  \end{block}
\end{frame}

\begin{frame}{Noise model}
  \begin{block}{}
  \begin{itemize}
    \item{
    Let's assume a simple model for the measurement noise: \\
    \bigskip
    Measurement noise in $x$ and measurement noise in $y$ are uncorrelated Gaussian with standard deviations $\sigma_x$ and $\sigma_y$ (which we will assume to both be unity).
    }
    \end{itemize}
  \end{block}
\end{frame}

\begin{frame}{Data}
  \begin{block}{}
  \begin{itemize}
    \item{We make one observation of $x$ (call the measured value $\tilde{x}$) and one of $y$ (call the measured value $\tilde{y}$). \\ \bigskip}
    \item{Assume that we are in the low signal-to-noise regime, so $\tilde{x}$ and $\tilde{y}$ are `a few' (e.g. not `a few thousand'). For example, $\tilde{x} = 3$ and $\tilde{y} = 4$.}
    \end{itemize}
  \end{block}
\end{frame}

\begin{frame}{Naive calculation}
  \begin{block}{}
    \begin{itemize}
    \item{So here's a naive calculation that appears reasonable to do:}
    \item{By subtracting the (unknown) noise from $(\tilde{x},\tilde{y})$ we get a Gaussian distribution of the true values $(x,y)$ that is centred on $(\tilde{x},\tilde{y})$. For each possible true $(x,y)$ there is a true $b=y/x$, hence we have a distribution of the true $b$.}
   \end{itemize}
  \end{block}
\end{frame}



\end{document}