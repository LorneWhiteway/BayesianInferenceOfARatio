% http://www.ctan.org/tex-archive/macros/latex/contrib/beamer/examples
% http://latex.artikel-namsu.de/english/beamer-examples.html

%\documentclass{beamer}
\documentclass[usenames,dvipsnames]{beamer}
\usepackage{amsmath}
\usepackage{amssymb}
\usepackage{bm}
\usepackage{fancybox, graphicx}
\usepackage{listings}
\usepackage{tikz} % Diagrams
\usepackage{color}
\usepackage{textcomp} % See https://tex.stackexchange.com/questions/145416/how-to-have-straight-single-quotes-in-lstlistings

\lstset{language=bash,upquote=true} % Format listings as appropriate for bash. Inexplicably we get problems if the language is set as part of the \begin{lstlisting} command.

% https://tex.stackexchange.com/questions/36030/how-to-make-a-single-word-look-as-some-code
\definecolor{light-gray}{gray}{0.95}
\newcommand{\code}[1]{\colorbox{light-gray}{\texttt{#1}}}

\usetheme{boxes}
\usecolortheme{beaver}

\title{What Can a Bayesian Say About $y/x$?}
\author{Lorne Whiteway \\ lorne.whiteway.13@ucl.ac.uk}
\institute{Astrophysics Group \\ Department of Physics and Astronomy \\ University College London}
\date{21 November 2025}

\begin{document}

\frame{\titlepage}

\begin{frame}{Purpose of presentation}
  \begin{block}{}
    \begin{itemize}
      \item{I'll show how to do Bayesian inference in a context that's simple but non-trivial.}
      \item{Talk will be mostly mathematical}
      \item{Existing knowledge of Bayesian ideas is useful but not necessary.}
    \end{itemize}
  \end{block}
\end{frame}

\begin{frame}{Problem}
  \begin{block}{}
    \begin{itemize}
      \item{We make noisy measurements of $x$ and $y$ and we want to infer $b = y/x$.}
      \item{The answer should be probabilistic (due to the uncertainty arising from the measurement noise).}
    \end{itemize}
  \end{block}
\end{frame}

\begin{frame}{Noise model}
  \begin{block}{}
    I want to keep the noise model simple, so let's asume: \\
    
    \setlength{\parindent}{1 cm} Noise in $x$ and noise in $y$ are uncorrelated Gaussian with standard deviations $\sigma_x$ and $\sigma_y$ (which we will assume to both be unity).
  \end{block}
\end{frame}

\begin{frame}{Data}
  \begin{block}{}
    We make one observation of $x$ (call the measured value $\tilde{x}$) and one of $y$ (call the measured value $\tilde{y}$).
    
    Assume that we are in the low signal-to-noise regime, so $\tilde{x}$ and $\tilde{y}$ are `a few' (e.g. not `a few thousand'). For example, $\tilde{x} = 3$ and $\tilde{y} = 4$.
    
  \end{block}
\end{frame}


\begin{frame}{Naive calculation}
  \begin{block}{}
    So here's a calculation that might appear reasonable to do.
  \end{block}
\end{frame}



\end{document}